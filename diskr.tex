\documentclass[]{article}

\usepackage[utf8x]{inputenc}
\usepackage[russian]{babel}
\usepackage{amsmath}
\usepackage{amssymb}
\usepackage[a4paper, total={7.5in, 10in}]{geometry}
\usepackage{array}
\newcolumntype{C}[1]{>{\centering\let\newline\\\arraybackslash\hspace{0pt}}m{#1}}
\newcommand{\tabitem}{~~\llap{\textbullet}~~}

\begin{document}
	\begin{figure}[t]
		\centering
		\fontseries{b}
		\large
		НАЦИОНАЛЬНЫЙ ИССЛЕДОВАТЕЛЬСКИЙ УНИВЕРСИТЕТ\\
		«ВЫСШАЯ ШКОЛА ЭКОНОМИКИ»\\
		Дисциплина: «Дискретная математика»
	\end{figure}
	
	\begin{figure}[h]
	\vspace{3in}
	\fontseries{b}
	\centering
	\Large
	Домашнее задание 1\\
	\Huge
	\textbf{Исследование комбинационных схем}\\
	Вариант 181 
	\end{figure}
	
	\vspace{2in}
	\Large
	\raggedleft
	Выполнил: Мартиросян Тигран Оганнесович,\\
	студент гр. 176\\
	\vspace{12pt}
	Преподаватель: Авдошин С.М.,\\
	профессор департамента\\
	программной инженерии\\
	факультета компьютерных наук
	
	\begin{figure}[b]
		\centering
		Москва \the\year
	\end{figure}
	
	\thispagestyle{empty}
	
	\newpage
	
	\Large
	\begin{center}№1\end{center}
	
	\begin{equation*}
	7X_7 \oplus 186X_6\oplus 213X_5\oplus 21X_4\oplus 238X_3 \oplus 142X_2\oplus 30X_1 \oplus 191X_0 = 183
	\end{equation*}
	
	\raggedright
	Переведем коэффициенты уравнения в двоичную систему счисления.\\
		
	$7_{10} = 00000111_2, 186_{10} = 10111010_2, 213_{10} = 11010101_2, 21_{10} = 00010101_2, 238_{10} = 11101110_2, 142_{10} = 10001110_2, 30_{10} = 00011110_2, 191_{10} = 10111111_2, 183_{10} = 10110111_2$.\\
		
	Составим расширенную матрицу коэффициентов соответствующей системы линейных уравнений в $GF(2)$ и решим систему. 
			
	\vspace{20pt}
	
	\small
	\centering
	\begin{tabular}{cccc}
		$\left(\begin{array}{cccccccc|c}
0&1&1&0&1&1&0&1&1\\
0&0&1&0&1&0&0&0&0\\
0&1&0&0&1&0&0&1&1\\
0&1&1&1&0&0&1&1&1\\
0&1&0&0&1&1&1&1&0\\
1&0&1&1&1&1&1&1&1\\
1&1&0&0&1&1&1&1&1\\
1&0&1&1&0&0&0&1&1\\\end{array}\right)$
&
\begin{tabular}{c}
$(6) \oplus= (5)$\\
$(7) \oplus= (5)$\\
\end{tabular}
$\left(\begin{array}{cccccccc|c}
0&1&1&0&1&1&0&1&1\\
0&0&1&0&1&0&0&0&0\\
0&1&0&0&1&0&0&1&1\\
0&1&1&1&0&0&1&1&1\\
0&1&0&0&1&1&1&1&0\\
1&0&1&1&1&1&1&1&1\\
0&1&1&1&0&0&0&0&0\\
0&0&0&0&1&1&1&0&0\\\end{array}\right)$
&
\begin{tabular}{c}
$(2) \oplus= (0)$\\
$(3) \oplus= (0)$\\
$(4) \oplus= (0)$\\
$(6) \oplus= (0)$\\
\end{tabular}
\\\\
$\left(\begin{array}{cccccccc|c}
0&1&1&0&1&1&0&1&1\\
0&0&1&0&1&0&0&0&0\\
0&0&1&0&0&1&0&0&0\\
0&0&0&1&1&1&1&0&0\\
0&0&1&0&0&0&1&0&1\\
1&0&1&1&1&1&1&1&1\\
0&0&0&1&1&1&0&1&1\\
0&0&0&0&1&1&1&0&0\\
\end{array}\right)$
&
\begin{tabular}{c}
$(0) \oplus= (1)$\\
$(2) \oplus= (1)$\\
$(4) \oplus= (1)$\\
$(5) \oplus= (1)$\\
\end{tabular}
$\left(\begin{array}{cccccccc|c}
0&1&0&0&0&1&0&1&1\\
0&0&1&0&1&0&0&0&0\\
0&0&0&0&1&1&0&0&0\\
0&0&0&1&1&1&1&0&0\\
0&0&0&0&1&0&1&0&1\\
1&0&0&1&0&1&1&1&1\\
0&0&0&1&1&1&0&1&1\\
0&0&0&0&1&1&1&0&0\\\end{array}\right)$
&
\begin{tabular}{c}
$(5) \oplus= (3)$\\
$(6) \oplus= (3)$\\
\end{tabular}
\\\\
$\left(\begin{array}{cccccccc|c}
0&1&0&0&0&1&0&1&1\\
0&0&1&0&1&0&0&0&0\\
0&0&0&0&1&1&0&0&0\\
0&0&0&1&1&1&1&0&0\\
0&0&0&0&1&0&1&0&1\\
1&0&0&0&1&0&0&1&1\\
0&0&0&0&0&0&1&1&1\\
0&0&0&0&1&1&1&0&0\\\end{array}\right)$
&
\begin{tabular}{c}
$(1) \oplus= (2)$\\
$(3) \oplus= (2)$\\
$(4) \oplus= (2)$\\
$(5) \oplus= (2)$\\
$(7) \oplus= (2)$\\
\end{tabular}
$\left(\begin{array}{cccccccc|c}
0&1&0&0&0&1&0&1&1\\
0&0&1&0&0&1&0&0&0\\
0&0&0&0&1&1&0&0&0\\
0&0&0&1&0&0&1&0&0\\
0&0&0&0&0&1&1&0&1\\
1&0&0&0&0&1&0&1&1\\
0&0&0&0&0&0&1&1&1\\
0&0&0&0&0&0&1&0&0\\\end{array}\right)$
&
\begin{tabular}{c}
$(0) \oplus= (4)$\\
$(1) \oplus= (4)$\\
$(2) \oplus= (4)$\\
$(5) \oplus= (4)$\\
\end{tabular}
\\\\
$\left(\begin{array}{cccccccc|c}
0&1&0&0&0&0&1&1&0\\
0&0&1&0&0&0&1&0&1\\
0&0&0&0&1&0&1&0&1\\
0&0&0&1&0&0&1&0&0\\
0&0&0&0&0&1&1&0&1\\
1&0&0&0&0&0&1&1&0\\
0&0&0&0&0&0&1&1&1\\
0&0&0&0&0&0&1&0&0\\\end{array}\right)$
&
\begin{tabular}{c}
$(0) \oplus= (6)$\\
$(1) \oplus= (6)$\\
$(2) \oplus= (6)$\\
$(3) \oplus= (6)$\\
$(4) \oplus= (6)$\\
$(5) \oplus= (6)$\\
$(7) \oplus= (6)$\\
\end{tabular}
$\left(\begin{array}{cccccccc|c}
0&1&0&0&0&0&0&0&1\\
0&0&1&0&0&0&0&1&0\\
0&0&0&0&1&0&0&1&0\\
0&0&0&1&0&0&0&1&1\\
0&0&0&0&0&1&0&1&0\\
1&0&0&0&0&0&0&0&1\\
0&0&0&0&0&0&1&1&1\\
0&0&0&0&0&0&0&1&1\\\end{array}\right)$
&
\begin{tabular}{c}
$(1) \oplus= (7)$\\
$(2) \oplus= (7)$\\
$(3) \oplus= (7)$\\
$(4) \oplus= (7)$\\
$(6) \oplus= (7)$\\
\end{tabular}
\\\\
$\left(\begin{array}{cccccccc|c}
0&1&0&0&0&0&0&0&1\\
0&0&1&0&0&0&0&0&1\\
0&0&0&0&1&0&0&0&1\\
0&0&0&1&0&0&0&0&0\\
0&0&0&0&0&1&0&0&1\\
1&0&0&0&0&0&0&0&1\\
0&0&0&0&0&0&1&0&0\\
0&0&0&0&0&0&0&1&1\\\end{array}\right)$
&
\\\\

	\end{tabular}
	
	\Large
	В описаниях преобразований строки обозначены как (1), (2), …, (8),\\
	а выражение (i)⊕=(j)  обозначает «заменить все числа в строке (i)\\
	на их сумму по модулю 2 с соответствующими числами строки (j)».
	
	\newpage
	
	\raggedright
	Получаем решение: X7 = 1, X6 = 1, X5 = 1, X4 = 0, X3 = 1, X2 = 1, X1 = 0, X0 = 1.\\
	Составим таблицу истинности функции F.\\
	\vspace{10pt}
	\centering
	\begin{tabular}{|c|c|c|c|c|c|c|c|c|}
		\hline
		A&0&0&0&0&1&1&1&1\\
		\hline
		B&0&0&1&1&0&0&1&1\\
		\hline
		C&0&1&0&1&0&1&0&1\\
		\hline
		F&1&1&1&0&1&1&0&1\\
		\hline
	\end{tabular}
	\vspace{10pt}
	
	\raggedright
	Десятичный номер функции F равен $2^7 + 2^6 + 2^5+ 2^3 + 2^2 + 2^0 = 237$.
	
	\vspace{20pt}
	\centering
	\normalsize
	\begin{center}\begin{large}{№2}\end{large}\end{center}
	\textbf{Представим таблицу истинности логической функции F в виде карты Карно.}
	
	\centering
	\normalsize
	
	
	\begin{tabular}{|c|c|c|c|c|c|}
	\hline
	&0&0&1&1&A \\
	\cline{2-6}
	\raisebox{1.5ex}[0cm][0cm]{F}
	&0&1&1&0&B \\
	\hline
	0&1&1&0&1 \\
	\cline{1-5}
	1&1&0&1&1\\
	\cline{1-5}
	C\\
	\cline{1-1}
	
\end{tabular}
	
\begin{center}\begin{large}{№3}\end{large}\end{center}
\textbf{Выполним дизъюнктивны разложения Шеннона логической функции F.}
	\[
\begin{aligned}
	\begin{pmatrix}
	A&0&0&0&0&1&1&1&1\\
	B&0&0&1&1&0&0&1&1\\
	C&0&1&0&1&0&1&0&1\\
	F&1&1&1&0&1&1&0&1\\
	\end{pmatrix}
	&=\overline{A}\cdot
	\begin{pmatrix}
	B&0&0&1&1\\
	C&0&1&0&1\\
	F&1&1&1&0\\
	\end{pmatrix}
	+A\cdot
	\begin{pmatrix}
	B&0&0&1&1\\
	C&0&1&0&1\\
	F&1&1&0&1\\
	\end{pmatrix}
	=\overline{A}\cdot (B\mid C)+A\cdot (B\Rightarrow C) 
	\\
	\begin{pmatrix}
	A&0&0&0&0&1&1&1&1\\
	B&0&0&1&1&0&0&1&1\\
	C&0&1&0&1&0&1&0&1\\
	F&1&1&1&0&1&1&0&1\\
	\end{pmatrix}
	&=\overline{B}\cdot
	\begin{pmatrix}
	A&0&0&1&1\\
	C&0&1&0&1\\
	F&1&1&1&1\\
	\end{pmatrix}
	+B\cdot
	\begin{pmatrix}
	A&0&0&1&1\\
	C&0&1&0&1\\
	F&1&0&0&1\\
	\end{pmatrix}
	=\overline{B}+B\cdot (A\equiv C) 
	\\
	\begin{pmatrix}
	A&0&0&0&0&1&1&1&1\\
	B&0&0&1&1&0&0&1&1\\
	C&0&1&0&1&0&1&0&1\\
	F&1&1&1&0&1&1&0&1\\
	\end{pmatrix}
	&=\overline{C}\cdot
	\begin{pmatrix}
	A&0&0&1&1\\
	B&0&1&0&1\\
	F&1&1&1&0\\
	\end{pmatrix}
	+C\cdot
	\begin{pmatrix}
	A&0&0&1&1\\
	B&0&1&0&1\\
	F&1&0&1&1\\
	\end{pmatrix}
	=\overline{C}\cdot (A\mid B)+C\cdot (A\Leftarrow B) 
\end{aligned}
	\]
	\begin{center}\begin{large}{№4}\end{large}\end{center}
	\textbf{Совершенная дизъюнктивная нормальная форма}
	\begin{equation}
	F(A,B,C)=\bar{A}\bar{B}\bar{C}+\bar{A}\bar{B}C+\bar{A}B\bar{C}+A\bar{B}\bar{C}+A\bar{B}C+ABC
	\end{equation}
	\begin{center}\begin{large}{№5}\end{large}\end{center}
	\textbf{Минимальные дизъюнктивные формы}
	\begin{equation}
	F(A,B,C)=\overline{B}+AC+\overline{A}\overline{C}
	\end{equation}
	\begin{center}\begin{large}{№6}\end{large}\end{center}
	
	\textbf{Из дизъюнктивных разложений получаем новые представления}
	\[
	\begin{aligned}
		F(A,B,C)&=\overline{A}\cdot (B\mid C)\oplus A\cdot (B\Rightarrow C) \\
		F(A,B,C)&=\overline{B}\oplus B\cdot (A\equiv C)\\
		F(A,B,C)&=\overline{C}\cdot (A\mid B)\oplus C\cdot (A\Leftarrow B) \\
		F(A,B,C)&=\bar{A}\bar{B}\bar{C}\oplus \bar{A}\bar{B}C\oplus\bar{A}B\bar{C}\oplus A\bar{B}\bar{C}\oplus A\bar{B}C\oplus ABC
	\end{aligned}
	\]
	
	\newpage
	\begin{center}\begin{large}{№7}\end{large}\end{center}
	\textbf{Выполним конъюнктивные разложения Шеннона логической функции F.}
	\[
	\begin{aligned}
		\begin{pmatrix}
			A&0&0&0&0&1&1&1&1\\
			B&0&0&1&1&0&0&1&1\\
			C&0&1&0&1&0&1&0&1\\
			F&1&1&1&0&1&1&0&1\\
		\end{pmatrix}
		&=\left(A+
		\begin{pmatrix}
			B&0&0&1&1\\
			C&0&1&0&1\\
			F&1&1&1&0\\
		\end{pmatrix}
		\right)
		\cdot \left(\overline{A}+
		\begin{pmatrix}
			B&0&0&1&1\\
			C&0&1&0&1\\
			F&1&1&0&1\\
		\end{pmatrix}\right)
		=(A + (B\mid C))\cdot (\overline{A} + (B\Rightarrow C) )
		\\
		\begin{pmatrix}
		A&0&0&0&0&1&1&1&1\\
		B&0&0&1&1&0&0&1&1\\
		C&0&1&0&1&0&1&0&1\\
		F&1&1&1&0&1&1&0&1\\
		\end{pmatrix}
		&=\left(B+
		\begin{pmatrix}
		A&0&0&1&1\\
		C&0&1&0&1\\
		F&1&1&1&1\\
		\end{pmatrix}
		\right)
		\cdot \left(\overline{B}+
		\begin{pmatrix}
		A&0&0&1&1\\
		C&0&1&0&1\\
		F&1&0&0&1\\
		\end{pmatrix}\right)
		=B\cdot(\overline{B}+ (A\equiv C)) 
		\\
		\begin{pmatrix}
		A&0&0&0&0&1&1&1&1\\
		B&0&0&1&1&0&0&1&1\\
		C&0&1&0&1&0&1&0&1\\
		F&1&1&1&0&1&1&0&1\\
		\end{pmatrix}
		&=\left(C+
		\begin{pmatrix}
		A&0&0&1&1\\
		B&0&1&0&1\\
		F&1&1&1&0\\
		\end{pmatrix}
		\right)
		\cdot \left(\overline{C}+
		\begin{pmatrix}
		A&0&0&1&1\\
		B&0&1&0&1\\
		F&1&0&1&1\\
		\end{pmatrix}\right)
		=(C+(A\mid B))\cdot (\overline{C}+(A\Leftarrow B))
	\end{aligned}
	\]
	\begin{center}\begin{large}{№8}\end{large}\end{center}
	\textbf{Совершенная конъюнктивная нормальная форма.}
	\begin{equation}
	F(A,B,C)= (A+\overline{B}+\overline{C})\cdot(\overline{A}+\overline{B}+C) 
	\end{equation}
	\begin{center}\begin{large}{№9}\end{large}\end{center}
	\textbf{Минимальные конъюнктивные нормальные формы.}
	\begin{equation}
	F(A,B,C)= (A+\overline{B}+\overline{C})\cdot(\overline{A}+\overline{B}+C) 
	\end{equation}
	
	
	
	\normalsize
	
	\begin{center}\begin{large}{№10}\end{large}\end{center}
	\textbf{Из конъюнктивных разложений, используя ортогональность,
	получаем новые представления функции}
	\[
	\begin{aligned}
	F(A,B,C) &= (A+(B\mid C))\equiv(\overline{A}+(B\Rightarrow C))\\
	F(A,B,C) &= B\equiv(\overline{B}+(A\equiv C))\\
	F(A,B,C) &= (C+(A\mid B)) \equiv (\overline{C}+(A\Leftarrow B))\\
	F(A,B,C) &= (A+\overline{B}+\overline{C})\equiv(\overline{A}+\overline{B}+C) 
	\end{aligned}
	\]
	
	
	\begin{center}\begin{large}{№11}\end{large}\end{center}
	\textbf{Вычислим производные логической функции F.}
	\[
	\begin{aligned}
		F^{'}_A(A,B,C)=
		\begin{pmatrix}
			A&0&0&0&0&1&1&1&1\\
			B&0&0&1&1&0&0&1&1\\
			C&0&1&0&1&0&1&0&1\\
			F&1&1&1&0&1&1&0&1\\
		\end{pmatrix}^{'}_A
		&=
		\begin{pmatrix}
			B&0&0&1&1\\
			C&0&1&0&1\\
			F&1&1&1&0\\
		\end{pmatrix}
		\oplus
		\begin{pmatrix}
			B&0&0&1&1\\
			C&0&1&0&1\\
			F&1&1&0&1\\
		\end{pmatrix}
		=
		\begin{pmatrix}
		B&0&0&1&1\\
		C&0&1&0&1\\
		F&0&0&1&1\\
		\end{pmatrix}
		=
		B 
		\\
		F^{'}_B(A,B,C)=
		\begin{pmatrix}
			A&0&0&0&0&1&1&1&1\\
			B&0&0&1&1&0&0&1&1\\
			C&0&1&0&1&0&1&0&1\\
			F&1&1&1&0&1&1&0&1\\
		\end{pmatrix}^{'}_B
		&=
		\begin{pmatrix}
			A&0&0&1&1\\
			C&0&1&0&1\\
			F&1&1&1&1\\
		\end{pmatrix}
		\oplus
		\begin{pmatrix}
			A&0&0&1&1\\
			C&0&1&0&1\\
			F&1&0&0&1\\
		\end{pmatrix}
		=
		\begin{pmatrix}
		A&0&0&1&1\\
		C&0&1&0&1\\
		F&0&1&1&0\\
		\end{pmatrix}
		=A+C
		\\
		F^{'}_C(A,B,C)=
		\begin{pmatrix}
			A&0&0&0&0&1&1&1&1\\
			B&0&0&1&1&0&0&1&1\\
			C&0&1&0&1&0&1&0&1\\
			F&1&1&1&0&1&1&0&1\\
		\end{pmatrix}^{'}_C
		&=
		\begin{pmatrix}
			A&0&0&1&1\\
			B&0&1&0&1\\
			F&1&1&1&0\\
		\end{pmatrix}
		\oplus
		\begin{pmatrix}
			A&0&0&1&1\\
			B&0&1&0&1\\
			F&1&0&1&1\\
		\end{pmatrix}
		=
		\begin{pmatrix}
		A&0&0&1&1\\
		B&0&1&0&1\\
		F&0&1&0&1\\
		\end{pmatrix}
		=B
	\end{aligned}
	\]
	\newpage
	
	\begin{center}\begin{large}{№12}\end{large}\end{center}
	\textbf{Разложения Рида логической функции F.}	
	\[
	\begin{aligned}
	F(A, B, C) &= F(0, B, C)\oplus A \cdot F^{'}_A(A,B,C)=(B\mid C)\oplus A \cdot B\\
	F(A, B, C) &= F(1, B, C)\oplus \overline{A} \cdot F^{'}_A(A,B,C)=(B\Rightarrow C)\oplus \overline{A}\cdot B\\
	F(A, B, C) &= F(A, 0, C)\oplus B\cdot F^{'}_B(A,B,C)=(1\oplus B)\cdot(A+C)\\
	F(A, B, C) &= F(A, 1, C)\oplus \overline{B} \cdot F^{'}_B(A,B,C)=(A\equiv C)\oplus\overline{B}\cdot(A+C)\\
	F(A, B, C) &= F(A, B, 0)\oplus C \cdot F^{'}_C(A,B,C)=(A\mid B)\oplus C \cdot B\\
	F(A, B, C) &= F(A, B, 1)\oplus \overline{C} \cdot F^{'}_C(A,B,C)=(A\Leftarrow B)\oplus\overline{C}\cdot B\\
	\end{aligned}
	\]
	
	\begin{center}\begin{large}{№13}\end{large}\end{center}
	\textbf{Двойственные разложения Рида логической функции F.}
	\[
	\begin{aligned}
	F(A, B, C) &= F(0, B, C)\equiv (\overline{A}+ \overline{F^{'}_A(A,B,C)})=(B\mid C)\equiv (\overline{A} + \overline{B})\\
	F(A, B, C) &= F(1, B, C)\equiv ({A} + \overline{F^{'}_A(A,B,C)})=(B\Rightarrow C)\equiv (A+ \overline{B})\\
	F(A, B, C) &= F(A, 0, C)\equiv (\overline{B}+ \overline{F^{'}_B(A,B,C)})=1\equiv (\overline{B}+(A\circ C)\\
	F(A, B, C) &= F(A, 1, C)\equiv (B + \overline{F^{'}_B(A,B,C)})=(A\equiv C)\equiv(B+(A\circ C)\\
	F(A, B, C) &= F(A, B, 0)\equiv (\overline{C} + \overline{F^{'}_C(A,B,C)})=(A\mid B)\equiv (\overline{C} + B)\\
	F(A, B, C) &= F(A, B, 1)\equiv (C + \overline{F^{'}_C(A,B,C)})=(A\Leftarrow B)\equiv (C+ B)\\
	\end{aligned}
	\]
	\begin{center}\begin{large}{№13}\end{large}\end{center}		\textbf{Смешанные производные в табличном и аналитическом виде.}
	\[
	\begin{aligned}
	\begin{tabular}{|c|c|c|c|c|c|c|c|c|}
			\hline
			A&0&0&0&0&1&1&1&1\\
			\hline
			B&0&0&1&1&0&0&1&1\\
			\hline
			C&0&1&0&1&0&1&0&1\\
			\hline
			F&1&1&1&0&1&1&0&1\\
			\hline
$F'_A$		 &&&&&&&&\\
\hline
$F'_B$	 	 &&&&&&&&\\
\hline
$F'_C$	 	 &&&&&&&&\\
\hline
$F''_{AB}$	 &&&&&&&&\\
\hline
$F''_{AC}$	 &&&&&&&&\\
\hline
$F''_{BC}$	 &&&&&&&&\\
\hline
$F'''_{ABC}$ &&&&&&&&\\
\hline
	\end{tabular}
	&\hspace{2cm}
	\begin{aligned}
	&=sss\\
	dfdfddd&ffggfd
	\end{aligned}
	\end{aligned}
	\]
\end{document}
\grid
