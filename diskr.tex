\documentclass[]{article}

\usepackage[utf8x]{inputenc}
\usepackage[russian]{babel}
\usepackage{amsmath}
\usepackage{amssymb}
\usepackage{graphicx}
\graphicspath{{basises/}}
\DeclareGraphicsExtensions{.png,.jpg}
\usepackage[a4paper, total={7.5in, 10in}]{geometry}
\usepackage{array}
\newcolumntype{C}[1]{>{\centering\let\newline\\\arraybackslash\hspace{0pt}}m{#1}}
\newcommand{\tabitem}{~~\llap{\textbullet}~~}

\begin{document}
	\begin{figure}[t]
		\centering
		\fontseries{b}
		\large
		НАЦИОНАЛЬНЫЙ ИССЛЕДОВАТЕЛЬСКИЙ УНИВЕРСИТЕТ\\
		«ВЫСШАЯ ШКОЛА ЭКОНОМИКИ»\\
		Дисциплина: «Дискретная математика»
	\end{figure}
	
	\begin{figure}[h]
	\vspace{3in}
	\fontseries{b}
	\centering
	\Large
	Домашнее задание 1\\
	\Huge
	\textbf{Исследование комбинационных схем}\\
	Вариант 181 
	\end{figure}
	
	\vspace{2in}
	\Large
	\raggedleft
	Выполнил: Мартиросян Тигран Оганнесович,\\
	студент гр. 176\\
	\vspace{12pt}
	Преподаватель: Авдошин С.М.,\\
	профессор департамента\\
	программной инженерии\\
	факультета компьютерных наук
	
	\begin{figure}[b]
		\centering
		Москва \the\year
	\end{figure}
	
	\thispagestyle{empty}
	
	\newpage
	
	\Large
	\begin{center}№1\end{center}
	
	\begin{equation*}
	7X_7 \oplus 186X_6\oplus 213X_5\oplus 21X_4\oplus 238X_3 \oplus 142X_2\oplus 30X_1 \oplus 191X_0 = 183
	\end{equation*}
	
	\raggedright
	Переведем коэффициенты уравнения в двоичную систему счисления.\\
		
	$7_{10} = 00000111_2, 186_{10} = 10111010_2, 213_{10} = 11010101_2, 21_{10} = 00010101_2, 238_{10} = 11101110_2, 142_{10} = 10001110_2, 30_{10} = 00011110_2, 191_{10} = 10111111_2, 183_{10} = 10110111_2$.\\
		
	Составим расширенную матрицу коэффициентов соответствующей системы линейных уравнений в $GF(2)$ и решим систему. 
			
	\vspace{20pt}
	
	\small
	\centering
	\begin{tabular}{cccc}
		$\left(\begin{array}{cccccccc|c}
0&1&1&0&1&1&0&1&1\\
0&0&1&0&1&0&0&0&0\\
0&1&0&0&1&0&0&1&1\\
0&1&1&1&0&0&1&1&1\\
0&1&0&0&1&1&1&1&0\\
1&0&1&1&1&1&1&1&1\\
1&1&0&0&1&1&1&1&1\\
1&0&1&1&0&0&0&1&1\\\end{array}\right)$
&
\begin{tabular}{c}
$(6) \oplus= (5)$\\
$(7) \oplus= (5)$\\
\end{tabular}
$\left(\begin{array}{cccccccc|c}
0&1&1&0&1&1&0&1&1\\
0&0&1&0&1&0&0&0&0\\
0&1&0&0&1&0&0&1&1\\
0&1&1&1&0&0&1&1&1\\
0&1&0&0&1&1&1&1&0\\
1&0&1&1&1&1&1&1&1\\
0&1&1&1&0&0&0&0&0\\
0&0&0&0&1&1&1&0&0\\\end{array}\right)$
&
\begin{tabular}{c}
$(2) \oplus= (0)$\\
$(3) \oplus= (0)$\\
$(4) \oplus= (0)$\\
$(6) \oplus= (0)$\\
\end{tabular}
\\\\
$\left(\begin{array}{cccccccc|c}
0&1&1&0&1&1&0&1&1\\
0&0&1&0&1&0&0&0&0\\
0&0&1&0&0&1&0&0&0\\
0&0&0&1&1&1&1&0&0\\
0&0&1&0&0&0&1&0&1\\
1&0&1&1&1&1&1&1&1\\
0&0&0&1&1&1&0&1&1\\
0&0&0&0&1&1&1&0&0\\
\end{array}\right)$
&
\begin{tabular}{c}
$(0) \oplus= (1)$\\
$(2) \oplus= (1)$\\
$(4) \oplus= (1)$\\
$(5) \oplus= (1)$\\
\end{tabular}
$\left(\begin{array}{cccccccc|c}
0&1&0&0&0&1&0&1&1\\
0&0&1&0&1&0&0&0&0\\
0&0&0&0&1&1&0&0&0\\
0&0&0&1&1&1&1&0&0\\
0&0&0&0&1&0&1&0&1\\
1&0&0&1&0&1&1&1&1\\
0&0&0&1&1&1&0&1&1\\
0&0&0&0&1&1&1&0&0\\\end{array}\right)$
&
\begin{tabular}{c}
$(5) \oplus= (3)$\\
$(6) \oplus= (3)$\\
\end{tabular}
\\\\
$\left(\begin{array}{cccccccc|c}
0&1&0&0&0&1&0&1&1\\
0&0&1&0&1&0&0&0&0\\
0&0&0&0&1&1&0&0&0\\
0&0&0&1&1&1&1&0&0\\
0&0&0&0&1&0&1&0&1\\
1&0&0&0&1&0&0&1&1\\
0&0&0&0&0&0&1&1&1\\
0&0&0&0&1&1&1&0&0\\\end{array}\right)$
&
\begin{tabular}{c}
$(1) \oplus= (2)$\\
$(3) \oplus= (2)$\\
$(4) \oplus= (2)$\\
$(5) \oplus= (2)$\\
$(7) \oplus= (2)$\\
\end{tabular}
$\left(\begin{array}{cccccccc|c}
0&1&0&0&0&1&0&1&1\\
0&0&1&0&0&1&0&0&0\\
0&0&0&0&1&1&0&0&0\\
0&0&0&1&0&0&1&0&0\\
0&0&0&0&0&1&1&0&1\\
1&0&0&0&0&1&0&1&1\\
0&0&0&0&0&0&1&1&1\\
0&0&0&0&0&0&1&0&0\\\end{array}\right)$
&
\begin{tabular}{c}
$(0) \oplus= (4)$\\
$(1) \oplus= (4)$\\
$(2) \oplus= (4)$\\
$(5) \oplus= (4)$\\
\end{tabular}
\\\\
$\left(\begin{array}{cccccccc|c}
0&1&0&0&0&0&1&1&0\\
0&0&1&0&0&0&1&0&1\\
0&0&0&0&1&0&1&0&1\\
0&0&0&1&0&0&1&0&0\\
0&0&0&0&0&1&1&0&1\\
1&0&0&0&0&0&1&1&0\\
0&0&0&0&0&0&1&1&1\\
0&0&0&0&0&0&1&0&0\\\end{array}\right)$
&
\begin{tabular}{c}
$(0) \oplus= (6)$\\
$(1) \oplus= (6)$\\
$(2) \oplus= (6)$\\
$(3) \oplus= (6)$\\
$(4) \oplus= (6)$\\
$(5) \oplus= (6)$\\
$(7) \oplus= (6)$\\
\end{tabular}
$\left(\begin{array}{cccccccc|c}
0&1&0&0&0&0&0&0&1\\
0&0&1&0&0&0&0&1&0\\
0&0&0&0&1&0&0&1&0\\
0&0&0&1&0&0&0&1&1\\
0&0&0&0&0&1&0&1&0\\
1&0&0&0&0&0&0&0&1\\
0&0&0&0&0&0&1&1&1\\
0&0&0&0&0&0&0&1&1\\\end{array}\right)$
&
\begin{tabular}{c}
$(1) \oplus= (7)$\\
$(2) \oplus= (7)$\\
$(3) \oplus= (7)$\\
$(4) \oplus= (7)$\\
$(6) \oplus= (7)$\\
\end{tabular}
\\\\
$\left(\begin{array}{cccccccc|c}
0&1&0&0&0&0&0&0&1\\
0&0&1&0&0&0&0&0&1\\
0&0&0&0&1&0&0&0&1\\
0&0&0&1&0&0&0&0&0\\
0&0&0&0&0&1&0&0&1\\
1&0&0&0&0&0&0&0&1\\
0&0&0&0&0&0&1&0&0\\
0&0&0&0&0&0&0&1&1\\\end{array}\right)$
&
\\\\

	\end{tabular}
	
	\Large
	В описаниях преобразований строки обозначены как (1), (2), …, (8),\\
	а выражение (i)⊕=(j)  обозначает «заменить все числа в строке (i)\\
	на их сумму по модулю 2 с соответствующими числами строки (j)».
	
	\newpage
	
	\raggedright
	Получаем решение: X7 = 1, X6 = 1, X5 = 1, X4 = 0, X3 = 1, X2 = 1, X1 = 0, X0 = 1.\\
	Составим таблицу истинности функции F.\\
	\vspace{10pt}
	\centering
	\begin{tabular}{|c|c|c|c|c|c|c|c|c|}
		\hline
		A&0&0&0&0&1&1&1&1\\
		\hline
		B&0&0&1&1&0&0&1&1\\
		\hline
		C&0&1&0&1&0&1&0&1\\
		\hline
		F&1&1&1&0&1&1&0&1\\
		\hline
	\end{tabular}
	\vspace{10pt}
	
	\raggedright
	Десятичный номер функции F равен $2^7 + 2^6 + 2^5+ 2^3 + 2^2 + 2^0 = 237$.
	
	\vspace{20pt}
	\centering
	\normalsize
	\begin{center}\begin{large}{№2}\end{large}\end{center}
	\textbf{Представим таблицу истинности логической функции F в виде карты Карно.}
	
	\centering
	\normalsize
	
	
	\begin{tabular}{|c|c|c|c|c|c|}
	\hline
	&0&0&1&1&A \\
	\cline{2-6}
	\raisebox{1.5ex}[0cm][0cm]{F}
	&0&1&1&0&B \\
	\hline
	0&1&1&0&1 \\
	\cline{1-5}
	1&1&0&1&1\\
	\cline{1-5}
	C\\
	\cline{1-1}
	
\end{tabular}
	
\begin{center}\begin{large}{№3}\end{large}\end{center}
\textbf{Выполним дизъюнктивны разложения Шеннона логической функции F.}
	\[
\begin{aligned}
	\begin{pmatrix}
	A&0&0&0&0&1&1&1&1\\
	B&0&0&1&1&0&0&1&1\\
	C&0&1&0&1&0&1&0&1\\
	F&1&1&1&0&1&1&0&1\\
	\end{pmatrix}
	&=\overline{A}\cdot
	\begin{pmatrix}
	B&0&0&1&1\\
	C&0&1&0&1\\
	F&1&1&1&0\\
	\end{pmatrix}
	+A\cdot
	\begin{pmatrix}
	B&0&0&1&1\\
	C&0&1&0&1\\
	F&1&1&0&1\\
	\end{pmatrix}
	=\overline{A}\cdot (B\mid C)+A\cdot (B\Rightarrow C) 
	\\
	\begin{pmatrix}
	A&0&0&0&0&1&1&1&1\\
	B&0&0&1&1&0&0&1&1\\
	C&0&1&0&1&0&1&0&1\\
	F&1&1&1&0&1&1&0&1\\
	\end{pmatrix}
	&=\overline{B}\cdot
	\begin{pmatrix}
	A&0&0&1&1\\
	C&0&1&0&1\\
	F&1&1&1&1\\
	\end{pmatrix}
	+B\cdot
	\begin{pmatrix}
	A&0&0&1&1\\
	C&0&1&0&1\\
	F&1&0&0&1\\
	\end{pmatrix}
	=\overline{B}+B\cdot (A\equiv C) 
	\\
	\begin{pmatrix}
	A&0&0&0&0&1&1&1&1\\
	B&0&0&1&1&0&0&1&1\\
	C&0&1&0&1&0&1&0&1\\
	F&1&1&1&0&1&1&0&1\\
	\end{pmatrix}
	&=\overline{C}\cdot
	\begin{pmatrix}
	A&0&0&1&1\\
	B&0&1&0&1\\
	F&1&1&1&0\\
	\end{pmatrix}
	+C\cdot
	\begin{pmatrix}
	A&0&0&1&1\\
	B&0&1&0&1\\
	F&1&0&1&1\\
	\end{pmatrix}
	=\overline{C}\cdot (A\mid B)+C\cdot (A\Leftarrow B) 
\end{aligned}
	\]
	\begin{center}\begin{large}{№4}\end{large}\end{center}
	\textbf{Совершенная дизъюнктивная нормальная форма}
	\begin{equation}
	F(A,B,C)=\bar{A}\bar{B}\bar{C}+\bar{A}\bar{B}C+\bar{A}B\bar{C}+A\bar{B}\bar{C}+A\bar{B}C+ABC
	\end{equation}
	\begin{center}\begin{large}{№5}\end{large}\end{center}
	\textbf{Минимальные дизъюнктивные формы}
	\begin{equation}
	F(A,B,C)=\overline{B}+AC+\overline{A}\hspace{1pt}\overline{C}
	\end{equation}
	\begin{center}\begin{large}{№6}\end{large}\end{center}
	
	\textbf{Из дизъюнктивных разложений получаем новые представления}
	\[
	\begin{aligned}
		F(A,B,C)&=\overline{A}\cdot (B\mid C)\oplus A\cdot (B\Rightarrow C) \\
		F(A,B,C)&=\overline{B}\oplus B\cdot (A\equiv C)\\
		F(A,B,C)&=\overline{C}\cdot (A\mid B)\oplus C\cdot (A\Leftarrow B) \\
		F(A,B,C)&=\bar{A}\bar{B}\bar{C}\oplus \bar{A}\bar{B}C\oplus\bar{A}B\bar{C}\oplus A\bar{B}\bar{C}\oplus A\bar{B}C\oplus ABC
	\end{aligned}
	\]
	
	\newpage
	\begin{center}\begin{large}{№7}\end{large}\end{center}
	\textbf{Выполним конъюнктивные разложения Шеннона логической функции F.}
	\[
	\begin{aligned}
		\begin{pmatrix}
			A&0&0&0&0&1&1&1&1\\
			B&0&0&1&1&0&0&1&1\\
			C&0&1&0&1&0&1&0&1\\
			F&1&1&1&0&1&1&0&1\\
		\end{pmatrix}
		&=\left(A+
		\begin{pmatrix}
			B&0&0&1&1\\
			C&0&1&0&1\\
			F&1&1&1&0\\
		\end{pmatrix}
		\right)
		\cdot \left(\overline{A}+
		\begin{pmatrix}
			B&0&0&1&1\\
			C&0&1&0&1\\
			F&1&1&0&1\\
		\end{pmatrix}\right)
		=(A + (B\mid C))\cdot (\overline{A} + (B\Rightarrow C) )
		\\
		\begin{pmatrix}
		A&0&0&0&0&1&1&1&1\\
		B&0&0&1&1&0&0&1&1\\
		C&0&1&0&1&0&1&0&1\\
		F&1&1&1&0&1&1&0&1\\
		\end{pmatrix}
		&=\left(B+
		\begin{pmatrix}
		A&0&0&1&1\\
		C&0&1&0&1\\
		F&1&1&1&1\\
		\end{pmatrix}
		\right)
		\cdot \left(\overline{B}+
		\begin{pmatrix}
		A&0&0&1&1\\
		C&0&1&0&1\\
		F&1&0&0&1\\
		\end{pmatrix}\right)
		=B\cdot(\overline{B}+ (A\equiv C)) 
		\\
		\begin{pmatrix}
		A&0&0&0&0&1&1&1&1\\
		B&0&0&1&1&0&0&1&1\\
		C&0&1&0&1&0&1&0&1\\
		F&1&1&1&0&1&1&0&1\\
		\end{pmatrix}
		&=\left(C+
		\begin{pmatrix}
		A&0&0&1&1\\
		B&0&1&0&1\\
		F&1&1&1&0\\
		\end{pmatrix}
		\right)
		\cdot \left(\overline{C}+
		\begin{pmatrix}
		A&0&0&1&1\\
		B&0&1&0&1\\
		F&1&0&1&1\\
		\end{pmatrix}\right)
		=(C+(A\mid B))\cdot (\overline{C}+(A\Leftarrow B))
	\end{aligned}
	\]
	\begin{center}\begin{large}{№8}\end{large}\end{center}
	\textbf{Совершенная конъюнктивная нормальная форма.}
	\begin{equation}
	F(A,B,C)= (A+\overline{B}+\overline{C})\cdot(\overline{A}+\overline{B}+C) 
	\end{equation}
	\begin{center}\begin{large}{№9}\end{large}\end{center}
	\textbf{Минимальные конъюнктивные нормальные формы.}
	\begin{equation}
	F(A,B,C)= (A+\overline{B}+\overline{C})\cdot(\overline{A}+\overline{B}+C) 
	\end{equation}
	
	
	
	\normalsize
	
	\begin{center}\begin{large}{№10}\end{large}\end{center}
	\textbf{Из конъюнктивных разложений, используя ортогональность,
	получаем новые представления функции}
	\[
	\begin{aligned}
	F(A,B,C) &= (A+(B\mid C))\equiv(\overline{A}+(B\Rightarrow C))\\
	F(A,B,C) &= B\equiv(\overline{B}+(A\equiv C))\\
	F(A,B,C) &= (C+(A\mid B)) \equiv (\overline{C}+(A\Leftarrow B))\\
	F(A,B,C) &= (A+\overline{B}+\overline{C})\equiv(\overline{A}+\overline{B}+C) 
	\end{aligned}
	\]
	
	
	\begin{center}\begin{large}{№11}\end{large}\end{center}
	\textbf{Вычислим производные логической функции F.}
	\[
	\begin{aligned}
		F^{'}_A(A,B,C)=
		\begin{pmatrix}
			A&0&0&0&0&1&1&1&1\\
			B&0&0&1&1&0&0&1&1\\
			C&0&1&0&1&0&1&0&1\\
			F&1&1&1&0&1&1&0&1\\
		\end{pmatrix}^{'}_A
		&=
		\begin{pmatrix}
			B&0&0&1&1\\
			C&0&1&0&1\\
			F&1&1&1&0\\
		\end{pmatrix}
		\oplus
		\begin{pmatrix}
			B&0&0&1&1\\
			C&0&1&0&1\\
			F&1&1&0&1\\
		\end{pmatrix}
		=
		\begin{pmatrix}
		B&0&0&1&1\\
		C&0&1&0&1\\
		F&0&0&1&1\\
		\end{pmatrix}
		=
		B 
		\\
		F^{'}_B(A,B,C)=
		\begin{pmatrix}
			A&0&0&0&0&1&1&1&1\\
			B&0&0&1&1&0&0&1&1\\
			C&0&1&0&1&0&1&0&1\\
			F&1&1&1&0&1&1&0&1\\
		\end{pmatrix}^{'}_B
		&=
		\begin{pmatrix}
			A&0&0&1&1\\
			C&0&1&0&1\\
			F&1&1&1&1\\
		\end{pmatrix}
		\oplus
		\begin{pmatrix}
			A&0&0&1&1\\
			C&0&1&0&1\\
			F&1&0&0&1\\
		\end{pmatrix}
		=
		\begin{pmatrix}
		A&0&0&1&1\\
		C&0&1&0&1\\
		F&0&1&1&0\\
		\end{pmatrix}
		=A\oplus C
		\\
		F^{'}_C(A,B,C)=
		\begin{pmatrix}
			A&0&0&0&0&1&1&1&1\\
			B&0&0&1&1&0&0&1&1\\
			C&0&1&0&1&0&1&0&1\\
			F&1&1&1&0&1&1&0&1\\
		\end{pmatrix}^{'}_C
		&=
		\begin{pmatrix}
			A&0&0&1&1\\
			B&0&1&0&1\\
			F&1&1&1&0\\
		\end{pmatrix}
		\oplus
		\begin{pmatrix}
			A&0&0&1&1\\
			B&0&1&0&1\\
			F&1&0&1&1\\
		\end{pmatrix}
		=
		\begin{pmatrix}
		A&0&0&1&1\\
		B&0&1&0&1\\
		F&0&1&0&1\\
		\end{pmatrix}
		=B
	\end{aligned}
	\]
	\newpage
	
	\begin{center}\begin{large}{№12}\end{large}\end{center}
	\textbf{Разложения Рида логической функции F.}	
	\[
	\begin{aligned}
	F(A, B, C) &= F(0, B, C)\oplus A \cdot F^{'}_A(A,B,C)=(B\mid C)\oplus A \cdot B\\
	F(A, B, C) &= F(1, B, C)\oplus \overline{A} \cdot F^{'}_A(A,B,C)=(B\Rightarrow C)\oplus \overline{A}\cdot B\\
	F(A, B, C) &= F(A, 0, C)\oplus B\cdot F^{'}_B(A,B,C)=(1\oplus B)\cdot(A\oplus C)\\
	F(A, B, C) &= F(A, 1, C)\oplus \overline{B} \cdot F^{'}_B(A,B,C)=(A\equiv C)\oplus\overline{B}\cdot(A\oplus C)\\
	F(A, B, C) &= F(A, B, 0)\oplus C \cdot F^{'}_C(A,B,C)=(A\mid B)\oplus C \cdot B\\
	F(A, B, C) &= F(A, B, 1)\oplus \overline{C} \cdot F^{'}_C(A,B,C)=(A\Leftarrow B)\oplus\overline{C}\cdot B\\
	\end{aligned}
	\]
	
	\begin{center}\begin{large}{№13}\end{large}\end{center}
	\textbf{Двойственные разложения Рида логической функции F.}
	\[
	\begin{aligned}
	F(A, B, C) &= F(0, B, C)\equiv (\overline{A}+ \overline{F^{'}_A(A,B,C)})=(B\mid C)\equiv (\overline{A} + \overline{B})\\
	F(A, B, C) &= F(1, B, C)\equiv ({A} + \overline{F^{'}_A(A,B,C)})=(B\Rightarrow C)\equiv (A+ \overline{B})\\
	F(A, B, C) &= F(A, 0, C)\equiv (\overline{B}+ \overline{F^{'}_B(A,B,C)})=1\equiv (\overline{B}+(A\equiv C)\\
	F(A, B, C) &= F(A, 1, C)\equiv (B + \overline{F^{'}_B(A,B,C)})=(A\equiv C)\equiv(B+(A\equiv C)\\
	F(A, B, C) &= F(A, B, 0)\equiv (\overline{C} + \overline{F^{'}_C(A,B,C)})=(A\mid B)\equiv (\overline{C} + B)\\
	F(A, B, C) &= F(A, B, 1)\equiv (C + \overline{F^{'}_C(A,B,C)})=(A\Leftarrow B)\equiv (C+ B)\\
	\end{aligned}
	\]
	\begin{center}\begin{large}{№14-15}\end{large}\end{center}		\textbf{Смешанные производные в табличном и аналитическом виде.}
	\[
	\begin{aligned}
	\begin{tabular}{|c|c|c|c|c|c|c|c|c|}
	\hline
A            & 0 & 0 & 0 & 0 & 1 & 1 & 1 & 1 \\ \hline
B            & 0 & 0 & 1 & 1 & 0 & 0 & 1 & 1 \\ \hline
C            & 0 & 1 & 0 & 1 & 0 & 1 & 0 & 1 \\ \hline
F            & 1 & 1 & 1 & 0 & 1 & 1 & 0 & 1 \\ \hline
$F'_A$       & 0 & 0 & 1 & 1 & 0 & 0 & 1 & 1 \\ \hline
$F'_B$       & 0 & 1 & 0 & 1 & 1 & 0 & 1 & 0 \\ \hline
$F'_C$       & 0 & 0 & 1 & 1 & 0 & 0 & 1 & 1 \\ \hline
$F''_{AB}$   & 1 & 1 & 1 & 1 & 1 & 1 & 1 & 1 \\ \hline
$F''_{AC}$   & 0 & 0 & 0 & 0 & 0 & 0 & 0 & 0 \\ \hline
$F''_{BC}$   & 1 & 1 & 1 & 1 & 1 & 1 & 1 & 1 \\ \hline
$F'''_{ABC}$ & 0 & 0 & 0 & 0 & 0 & 0 & 0 & 0 \\ \hline
	\end{tabular}
	&\hspace{2cm}
	\begin{aligned}
	F'_A&=B\\
	F'_B&=A\oplus C\\
	F'_C&=B\\
	F''_{AB}&=1\\
	F''_{AC}&=0\\
	F''_{BC}&=1\\
	F'''_{ABC}&=0
	\end{aligned}
	\end{aligned}
	\]
	\begin{center}\begin{large}{№16}\end{large}\end{center}
	\textbf{Разложим функцию F в ряд Маклорена в базисе $\{1,\oplus,\cdot\}$.\\
		Получим полином Жегалкина.}
	\begin{equation}
		F(A,B,C)=1\oplus BC\oplus AB
	\end{equation}
	
	\begin{center}\begin{large}{№17}\end{large}\end{center}
	\textbf{Разложим функцию F в ряд Тейлора в базисе $\{1,\oplus,\cdot\}$.\\}
	При разложении будем использовать сокращенную запись $\overline{A}=A\oplus 1,\overline{B} = B\oplus1,\overline{C}=C\oplus1.$
	\begin{align*}
	(0,0,0):F(A,B,C)&=1\oplus BC\oplus AB\\
	(0,0,1):F(A,B,C)&=1\oplus B\oplus AB\oplus B\overline{C}\\
	(0,1,0):F(A,B,C)&=1\oplus A\oplus C\oplus A\overline{B}\oplus \overline{B}C\\
	(0,1,1):F(A,B,C)&=A\oplus \overline{B}\oplus \overline{C}\oplus A\overline{B}\oplus\overline{B} \hspace{1,5pt} \overline{C}\\
	(1,0,0):F(A,B,C)&=1\oplus B\oplus\overline{A}B\oplus BC\\
	(1,0,1):F(A,B,C)&=1\oplus\overline{A}B\oplus B\overline{C}\\
	(1,1,0):F(A,B,C)&=\overline{A}\oplus\overline{B}\oplus C\oplus\overline{A}\hspace{1,5pt}\overline{B}\oplus\overline{B}C\\
	(1,1,1):F(A,B,C)&=1\oplus\overline{A}\oplus\overline{C}\oplus\overline{A}\hspace{1,5pt}\overline{B}\oplus\overline{B}\hspace{1,5pt}\overline{C}\\
	\end{align*}
	\newpage
	\begin{center}\begin{large}{№18}\end{large}\end{center}
	\textbf{Разложим функцию F в ряд Маклорена в базисе $\{0,\equiv,+\}$.}
	\begin{equation}
	F(A,B,C)=B\equiv(A+C)\equiv(A+B+C)
	\end{equation}
	\begin{center}\begin{large}{№19}\end{large}\end{center}
	\textbf{Разложим функцию F в ряд Тейлора в базисе $\{0,\equiv,+\}$.\\}
	При разложении будем использовать сокращенную запись $\overline{A}=A\equiv 0,\overline{B} = B\equiv0,\overline{C}=C\equiv0.$
	\begin{align*}
	(1,1,1):F(A,B,C)&=B\equiv(A+C)\equiv(A+B+C)\\
	(1,1,0):F(A,B,C)&=0\equiv(A+\overline{C})\equiv(A+B+\overline C)\\
	(1,0,1):F(A,B,C)&=A\equiv\overline{B}\equiv C\equiv(A+C)\equiv(A+\overline{B}+C)\\
	(1,0,0):F(A,B,C)&=A\equiv\overline{C}\equiv(A+\overline{C})\equiv(A+\overline{B}+\overline{C})\\
	(0,1,1):F(A,B,C)&=0\equiv(\overline{A}+C)\equiv(\overline{A}+B+C)\\
	(0,1,0):F(A,B,C)&=B\equiv(\overline{A}+\overline{C})\equiv(\overline{A}+B+\overline{C})\\
	(0,0,1):F(A,B,C)&=\overline{A}\equiv C\equiv(\overline{A}+C)\equiv(\overline{A}+\overline{B}+C)
	\\
	(0,0,0):F(A,B,C)&=\overline{A}\equiv\overline{B}\equiv\overline{C}\equiv(\overline{A}+\overline{C})\equiv(\overline{A}+\overline{B}+\overline{C})\\
	\end{align*}
	\begin{center}\begin{large}{№20}\end{large}\end{center}
	\textbf{Выразим функцию F с помощью минимального количества операций.}
	
	Из минимальной дизъюнктивной нормальной формы полученной в №9, выразим функцию F  с помощью двух операций.
	\[
	F(A,B,C)=\overline{B} + \overline{A}\hspace{1pt}\overline{C}+AC=\overline{B} + (A\equiv C)=B\Rightarrow(A\equiv C)
	\]
	Используя законы двойного отрицания и де Моргана, получаем:
	\[
	F(A,B,C)=\overline{B} + \overline{A}\hspace{1pt}\overline{C}+AC=\overline{\overline{\overline{B}+(A\equiv C)}}=\overline{B\cdot(A\oplus C)}=B\mid(A\oplus C)
	\]
	Для реализации в пограмме Winlogica комбинационной схемы используем первое полученно аналитическое представление функции F.
	\begin{figure}[h]
		\centering
		\includegraphics{18100}
	\end{figure}
	\newpage
	\begin{center}\begin{large}{№21}\end{large}\end{center}
	\textbf{Получим аналитические представления функции F в каждом из семнадцати базисов, соответствующие минимальному количеству блоков комбинационной схемы.}

	21.1
	
	21.2 В базисе $\{\mid\}$ отрицание можно выразить одним блоком, объединив два входа в один.
	Будем преобразовывать минимальную дизъюнктивную форму, полученную в №5.
	\begin{multline*}
	F(A,B,C)=\overline{B} + \overline{A}\hspace{1pt}\overline{C}+AC=\overline{CA+\overline{C}\hspace{1pt}\overline{A}}\mid B=\overline{(CA+\overline{C})\cdot(CA+\overline{A})}\mid B=\\
	=((\overline{\overline{CA+\overline{C}}})\mid(\overline{\overline{CA+\overline{A}}}))\mid B=
	((\overline{\overline{CA}\cdot C})\mid (\overline{\overline{CA}\cdot A)})\mid B=
	((((C\mid A) \mid C) \mid ((C\mid A)\mid A))\mid B)
	\end{multline*}
	\begin{figure}[h]
		\centering
		\includegraphics{18102}
	\end{figure}

	\newpage
	
	21.3 Найдем аналитическое представление функции F в базисе $\{0,\Rightarrow\}$.
	Будем преобразовывать минимальную дизъюнктивную форму, полученную в №5. 
	\begin{multline*}
	F(A,B,C)=\overline{B} +AC+ \overline{A}\hspace{1pt}\overline{C}=\overline{B} +(A\equiv C)=B\Rightarrow(\overline{\overline{(\overline{A}+C)\cdot(A+\overline{C})}}+0)=B\Rightarrow((\overline{(\overline{A}+C)\cdot(A+\overline{C})})\Rightarrow 0)=\\
	=B\Rightarrow(((\overline{\overline{A}+C})+(\overline{A+\overline{C}}))\Rightarrow 0)=B\Rightarrow (((A\Rightarrow C)\Rightarrow((C\Rightarrow A)\Rightarrow 0))\Rightarrow 0)
	\end{multline*}
	
	\begin{figure}[h!]
		\centering
		\includegraphics{18103}
	\end{figure}

	\newpage
	21.4
	
	\newpage
	21.5 Найдем аналитическое представление функции F в базисе $\{\Rightarrow,\nRightarrow\}$.
	
	Будем преобразовывать минимальную дизъюнктивную форму, полученную в №5.
	\[
		F(A,B,C)=\overline{B} +AC+ \overline{A}\hspace{1pt}\overline{C}=B\Rightarrow((\overline{A}+C)\cdot(\overline{C\cdot \overline{A}}))=B\Rightarrow((A\Rightarrow C)\nRightarrow(C\nRightarrow A))
	\] 
	\begin{figure}[h!]
	\centering
	\includegraphics{18105}
	\end{figure}

	21.6 Найдем аналитическое представление функции F в базисе $\{\oplus,\Rightarrow\}$.
	
	Будем преобразовывать минимальную дизъюнктивную форму, полученную в №5.
	\[
	F(A,B,C)=\overline{B} +AC+ \overline{A}\hspace{1pt}\overline{C}=B\Rightarrow(A\equiv C)
	=B\Rightarrow(\overline{A\oplus C})=(B\Rightarrow(\overline{A\oplus C}))\oplus B
	\] 
	\begin{figure}[h!]
		\centering
		\includegraphics{18106}
	\end{figure}

	\newpage
	21.7 Найдем аналитическое представление функции F в базисе $\{\equiv,\nRightarrow\}$.
	
	Будем преобразовывать минимальную дизъюнктивную форму, полученную в №5.
	\newpage
	
	21.8 Найдем аналитическое представление функции F в базисе $\{\neg,\Rightarrow\}$.
	Будем преобразовывать минимальную дизъюнктивную форму, полученную в №5.
	\[
		F(A,B,C)=\overline{B} + AC + \overline{A}\hspace{1pt}\overline{C}=
		\overline{B} + \overline{\overline{\overline{A}\hspace{1pt}\overline{C}}}+ AC=
		\overline{B}+(\overline{\overline{C}\hspace{1pt}\overline{A}}\Rightarrow CA)=
		B\Rightarrow((C+A)\Rightarrow \overline{\overline{CA}})=B\Rightarrow((\overline{C}\Rightarrow A)\Rightarrow (\overline{C\Rightarrow\overline{A}}))
	\]
	\begin{figure}[h!]
		\centering
		\includegraphics{18108}
	\end{figure}
	\newpage
	\begin{center}\begin{large}{№22-23}\end{large}\end{center}
	\textbf{Условия переключения сигнала на выходе схемы, реализующей функцию F, при переключении сигналов на каждой паре ее входов, в табличном и аналитическом виде.}
	\[
	\begin{aligned}
	\begin{tabular}{|c|c|c|c|c|c|c|c|c|}
	\hline
	A              & 0 & 0 & 0 & 0 & 1 & 1 & 1 & 1 \\ \hline
	B              & 0 & 0 & 1 & 1 & 0 & 0 & 1 & 1 \\ \hline
	C              & 0 & 1 & 0 & 1 & 0 & 1 & 0 & 1 \\ \hline
	F              & 1 & 1 & 1 & 0 & 1 & 1 & 0 & 1 \\ \hline
	$F'_{(A,B)}$   & 1 & 0 & 0 & 1 & 0 & 1 & 1 & 0 \\ \hline
	$F'_{(A,C)}$   & 0 & 0 & 0 & 0 & 0 & 0 & 0 & 0 \\ \hline
	$F'_{(B,C)}$   & 1 & 0 & 0 & 1 & 0 & 1 & 1 & 0 \\ \hline
	\end{tabular}
	\begin{aligned}
	\hspace{2cm}
	F'_{A,B}&=A\oplus(B\equiv C)\\
	F'_{A,C}&=0\\
	F'_{B,C}&=A\oplus(B\equiv C)\\
	\end{aligned}
	\end{aligned}
	\]
	\begin{center}\begin{large}{№24-25}\end{large}\end{center}
	\textbf{Условия переключения сигнала на выходе схемы, реализующей функцию F, при переключении сигналов на всех ее входах, в табличном и аналитическом виде.}
	\[
	\begin{aligned}
	\begin{tabular}{|c|c|c|c|c|c|c|c|c|}
	\hline
	A              & 0 & 0 & 0 & 0 & 1 & 1 & 1 & 1 \\ \hline
	B              & 0 & 0 & 1 & 1 & 0 & 0 & 1 & 1 \\ \hline
	C              & 0 & 1 & 0 & 1 & 0 & 1 & 0 & 1 \\ \hline
	F              & 1 & 1 & 1 & 0 & 1 & 1 & 0 & 1 \\ \hline
	$F'_{(A,B,C)}$ & 0 & 1 & 0 & 1 & 1 & 0 & 1 & 0 \\ \hline
	\end{tabular}
	\begin{aligned}
	\hspace{2cm}
	F'_{A,B,C}&=A\oplus C\\
	\end{aligned}
	\end{aligned}
	\]
	\begin{center}\begin{large}{№26}\end{large}\end{center}
	\textbf{Условия переключения сигнала на выходе схемы, реализующей функцию F, при переключении сигналов на всех ее входах, в табличном и аналитическом виде.}
	\[
	\begin{aligned}
	F&\notin T_0,\text{ т.к. } F(0,0,0)=1\\
	F&\in T_1,\text{ т.к. } F(1,1,1)=1\\
	F&\notin T_*\text{ т.к. } F(0,0,0)=F(1,1,1)=1\\
	F&\notin T_{\le} \text{ т.к. } F(1,0,0)=1>F(1,1,0)=0\\
	F&\notin T_{L} \text{ т.к.} F(A,B,C)=1\oplus AB\oplus BC \text{ содержит нелинейные члены}
	\end{aligned}
	\]
	\begin{center}\begin{large}{№27}\end{large}\end{center}
	\textbf{Выразим через функцию F функции двух переменных.}
	Поскольку $F\in F_1$ то из F c помощью операции суперпозиции нельзя получить никакие функции двух переменных $G\notin T_1$ то есть нельзя получить функции для которых $G(0,0)=1$.
	\[
	\begin{aligned}
	A\cdot B &= F(A,F(A,A,A),F(A,A,B))\\
	A+B&=F(A,F(A,B,B),F(A,B,B))\\
	A\equiv B &=F(A,F(A,A,A),B)\\
	A\Leftarrow B &= F(A,B,B)\\
	A\Rightarrow B&=F(A,A,B)\\
	1&=F(A,A,A)
	\end{aligned}
	\]
\end{document}
